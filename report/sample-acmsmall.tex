%%
%% This is file `sample-acmsmall.tex',
%% generated with the docstrip utility.
%%
%% The original source files were:
%%
%% samples.dtx  (with options: `acmsmall')
%% 
%% IMPORTANT NOTICE:
%% 
%% For the copyright see the source file.
%% 
%% Any modified versions of this file must be renamed
%% with new filenames distinct from sample-acmsmall.tex.
%% 
%% For distribution of the original source see the terms
%% for copying and modification in the file samples.dtx.
%% 
%% This generated file may be distributed as long as the
%% original source files, as listed above, are part of the
%% same distribution. (The sources need not necessarily be
%% in the same archive or directory.)
%%
%% The first command in your LaTeX source must be the \documentclass command.
 \documentclass[acmsmall]{acmart}

%%
%% \BibTeX command to typeset BibTeX logo in the docs
%%\AtBeginDocument{%
 %% \providecommand\BibTeX{{%
  %%  \normalfont B\kern-0.5em{\scshape i\kern-0.25em b}\kern-0.8em\TeX}}}

%% Rights management information.  This information is sent to you
%% when you complete the rights form.  These commands have SAMPLE
%% values in them; it is your responsibility as an author to replace
%% the commands and values with those provided to you when you
%% complete the rights form.



%%
%% These commands are for a JOURNAL article.
%% \acmJournal{JACM}
%% \acmVolume{37}
%% \acmNumber{4}
%% \acmArticle{111}

%%
%% Submission ID.
%% Use this when submitting an article to a sponsored event. You'll
%% receive a unique submission ID from the organizers
%% of the event, and this ID should be used as the parameter to this command.
%%\acmSubmissionID{123-A56-BU3}

%%
%% The majority of ACM publications use numbered citations and
%% references.  The command \citestyle{authoryear} switches to the
%% "author year" style.
%%
%% If you are preparing content for an event
%% sponsored by ACM SIGGRAPH, you must use the "author year" style of
%% citations and references.
%% Uncommenting
%% the next command will enable that style.
%%\citestyle{acmauthoryear}

%%
%% end of the preamble, start of the body of the document source.
\begin{document}

%%
%% The "title" command has an optional parameter,
%% allowing the author to define a "short title" to be used in page headers.
\title{Progetto 1 [SABD]}

%%
%% The "author" command and its associated commands are used to define
%% the authors and their affiliations.
%% Of note is the shared affiliation of the first two authors, and the
%% "authornote" and "authornotemark" commands
%% used to denote shared contribution to the research.
\author{Damiano Nardi}

\email{damiano6276@gmail.com}
\affiliation{%
 \institution{TorVergata, Corso Di Informatica}
 }

%%
%% By default, the full list of authors will be used in the page
%% headers. Often, this list is too long, and will overlap
%% other information printed in the page headers. This command allows
%% the author to define a more concise list
%% of authors' names for this purpose.

%%
%% The abstract is a short summary of the work to be presented in the

%\begin{abstract}

%\end{abstract}
%% article.

%%
%% The code below is generated by the tool at %%http://dl.acm.org/ccs.cfm.
%% Please copy and paste the code instead of the example below.
%%



%%
%% Keywords. The author(s) should pick words that accurately describe
%% the work being presented. Separate the keywords with commas.



%%
%% This command processes the author and affiliation and title
%% information and builds the first part of the formatted document.dsd
\maketitle{Progett 1 [SABD]} 

\section{Introduzione}
In questa relazione si descriverà il lavoro svolto che consiste nella realizzazione delle query(1,2) e tutta l'infrastruttura  composta da vari framework istanziati su container docker per l'ingest, process e store dei dati.


\section{Processamento Delle Query}
Per processare le entrambe le query è stato usato il framework Spark usando le api java.

\subsection{Query1}
\begin{quote}
Per ogni settimana, calcolare il numero medio di guariti e dei tamponi effettuati in Italia in quella
settimana.\end{quote}

Una volta caricato il dataset come JavaRDD viene fatta una 

\subsubsection{flatMapToPair} 
in cui viene restituito 
come chiave il numero della settimana dell'anno e come valore una tupla composta da positivi,tamponi (vengono ignorati i giorni in mezzo alla settimana in base allo startingDay selezionato) successivamente viene fatto

\subsubsection{reduceByKey}
dove viene calcolata la media dei tamponi e guariti facendo la differenza dei giorni di inizio e fine settimana.
\subsubsection{ricreazione schema}
Viene poi ricreato lo schema e viene salvato il dataset processato in formato parquet su HDFS

\subsubsection{Tempi}
è stata tenuta traccia di due tempi: tempo di processamento della query e tempo del caricamento dello spark contex.

\hspace{40mm} \begin{tabular}{l|l|}
\cline{2-2}
                                       & tempo (sec.) \\ \hline
\multicolumn{1}{|l|}{Query processing} & 3            \\ \hline
\multicolumn{1}{|l|}{Spark loading}    & 5            \\ \hline
\end{tabular}

\subsection{Query2}
\begin{quote}
Per ogni continente, calcolare la media, la deviazione standard, il minimo e il massimo del numero di
casi confermati giornalmente per ogni settimana. Nel calcolo delle statistiche, considerare solo i 100
stati piu colpiti dalla pandemia. Qualora lo stato non fosse indicato, considerare la nazione. Per de- `
terminare gli stati piu colpiti nell’intero dataset, si consideri l’andamento degli incrementi giornalieri `
dei casi confermati attraverso il trendline coefficient. Per stimare il trendline coefficient, si calcoli la
pendenza della retta di regressione che approssima la tendenza degli incrementi giornalieri.
Nota: il continente a cui appartiene ogni nazione non viene indicato in modo esplicito nel dataset, ma
deve essere ricavato. Si considerino 6 continenti: Africa, America, Antartide, Asia, Europa, Oceania.\end{quote}

Una volta caricato il dataset come JavaRDD viene fatta una 
\subsubsection{flatMapToPair}
in cui la chiave è il coefficente di di tread line e il valore è un oggeto
"CovidGlob",all'interno ha la lista dei contagiati, la regione e la nazione, durante questa fase viene calcolato il coefficente di di tread line, la lista degli infetti per giorno viene trasformata da cumulativa alla lista dei nuovi casi per ogni giorno.
\subsubsection{top 100}
Vegono presi i top 100 stati con il coefficente di tread line più alto

\subsubsection{mapToPair}
La chiave vine cambiata dal coefficente di tread line alla nazione il valore rimane l'oggetto  "CovidGlob"

\subsubsection{Join}
Viene caricato come javaRDD un dataset esterno al progetto da HDFS preso da \\ \url{https://github.com/dbouquin/IS_608/blob/master/NanosatDB_munging/Countries-Continents.csv}   
essenziamelte è mapping continete-nazione, viene fatto un a mapToPair per rendere la nazione chiave e poi viene fatto il join con i top 100

\subsubsection{mapToPair}
Per rendere il continente key il valore invece è la lista degli infetti per giorno 
\subsubsection{reduceByKey}
per sommare le liste di infetti con lo stesso continente

\subsubsection{flatMap}
Per calcolare tutte le statistiche richieste per settimana, viene usata la flatMap in quanto da ogni RDD element verranno generate 4 liste:
lista delle medie,massimi,minimi,deviazioni standard tutte quante per settimana 
\subsubsection{ricreazione schema}
Viene poi ricreato lo schema e viene salvato il dataset processato in formato parquet su HDFS

\subsubsection{Tempi}
è stata tenuta traccia di due tempi: tempo di processamento della query e tempo del caricamento dello spark contex.

\hspace{40mm} \begin{tabular}{l|l|}
\cline{2-2}
                                       & tempo (sec.) \\ \hline
\multicolumn{1}{|l|}{Query processing} & 3            \\ \hline
\multicolumn{1}{|l|}{Spark loading}    & 5            \\ \hline
\end{tabular}


\section{INFRASTRUTTURA DI PROCESSAMENTO}


\subsection{Framework utilizzati}

\subsubsection{Spark: processing} 
come visto prima si è utilizzato Spark per il processamento delle query 
\subsubsection{NIFI: ingestion}
Si è utilizzato NiFi per fare data ingestion, nello specifico i dati vengono scaricati da gitHub ogni settimana, sono scremati attraverso una query sql, convertiti in formato parquet e caricati su HDFS (questo per la query1, Nifi non è riuscito a convertire in parquet il dataset della query2 quindi è stato caricato direttamente il csv).
Di default Nifi non mette a disposizione un modo per comunicare con Spark, Livy viene in nostro (mio) aiuto permettendo a Nifi di inviare job a Spark e di sapere lo stato della computazione, in questo modo Nifi dopo aver finito l'upload dei dati su HDFS puo far iniziare a Spark il processamento e una volta finito Nifi può passare i dati processati al livello di storage
\subsubsection{Livy: layer di comunicazione}
Livy è un framework che mette a disposizione delle REST API per gestire Spark. In questo progetto è stato usato per far comunicare Nifi con Spark, nello specifico Nifi fa una post con dentro la classe main e la locazione del file jar nell' HDFS  da eseguire;
successivamente NiFi ogni 5 secondi controlla con una get a Livy lo stato del JoB, quando lo stato risulterà "succeed" Nifi continuerà il suo flusso 

\subsubsection{Hbase: Storage}
Per il layer di storage è stato utilizzato Hbase una volta che Spark ha finito la computazione Nifi si occupa di prendere i file parquet salvati 
da spark su HDFS processarli come parquet specificare la colonna che si vuole come key e caricarli su hbase.
Per la query1 come key ho scelto la data (formato annoSettimana)in quanto considerando la quantità dei dati idealmente si vuole la tabella sullo stesso regionserver in modo tale da ottimizzare lo scan di tutta la tabella (query più probabbile che avvenga). Mentre per la seconda query abbiamo righe di questo tipo :
\\
\\
\hspace{1000mm}\begin{tabular}{|l|l|l|l|}
\hline
AAAEuropa  & europa  & MAX\_WEEK & ... \\ \hline
ZZZamerica & america & MIN\_WEEK & ... \\ \hline
IIIEuropa  & europa  & DEV\_WEEK & ... \\ \hline
RRREuropa  & europa  & AVG\_WEEK & ... \\ \hline
\end{tabular}
\\
\\
\\
 come key è stato scelto il campo della prima colonna in questo modo le righe con la stessa funzione statica si troveranno sullo stesso regionserver e vicine tra loro andando ad ottimizare lo scan di tutti i continenti con una determinata funzione statistica 













\end{document}
\endinput
%%
%% End of file `sample-acmsmall.tex'.
